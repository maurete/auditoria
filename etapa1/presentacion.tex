\documentclass[spanish]{beamer}


% esto me setea la variable pdf dependiendo del valor de \pdfoutput, que es >0
% sólo cuando estoy usando pdflatex para compilar el documento
\newif\ifpdf
\ifnum\pdfoutput<0
\pdffalse\fi
\ifnum\pdfoutput=0
\pdffalse\fi
\ifnum\pdfoutput>0
\pdftrue\fi


%
% ===
% === Trick para detectar si el documento está siendo compilado con pdflatex
% ===
%
% Esto me setea la variable pdf dependiendo del valor de \pdfoutput, que es >0
% sólo cuando estoy usando pdflatex para compilar el documento. Con esto puedo
% hacer  \ifpdf {...} \fi, que se ejecuta colo cuando compilo con pdflatex.
\newif\ifpdf
\ifnum\pdfoutput<0
\pdffalse\fi
\ifnum\pdfoutput=0
\pdffalse\fi
\ifnum\pdfoutput>0
\pdftrue\fi
%
% ===
% === I18n / L10n
% ===
%
% babel me da separación de sílabas para palabras en el idioma que le paso como
%       argumento opcional.
\usepackage[spanish]{babel}
%
% inputenc define la codificación de caracteres del código fuente, acá utf8.
\usepackage[utf8]{inputenc}
%
% ===
% === Gráficos
% ===
% 
% pst-pdf me permite usar PSTricks con pdflatex. Necesito cargarlo sólo si está
%         definida la variable pdf, por eso está entre \ifpdf ... \fi
\ifpdf\usepackage{pst-pdf}\fi
%
% color me permite usar colores en el documento.
\usepackage{color}
%
% graphicx me da el comando \includegraphics para insertar imágenes (?)
\usepackage{graphicx}
%
% pstricks es un conjunto de macros basadas en PostScript para TeX, en
%          castellano: me da un entorno pstricks y comandos que uso dentro de
%          éste, que me sirven para dibujar figuras/diagramas/etc de manera
%          relativamente simple.
\usepackage{pstricks}
%
% pst-circ me da macros para pstricks que me dibujan elementos de circuitos
%\usepackage{pst-circ}
%
% pst-plot me provee de funciones de ploteo para pstricks
%\usepackage{pst-plot}
%
% pst-2dplot me sirve para plotear en pstricks, entorno pstaxes
%\usepackage{pst-2dplot}
%
% ===
% === Verbatims
% ===
%
% verbatim es una reimplementación de los entornos verbatim[*]
%          provee el comando \verbatiminput{archivo} y el entorno comment, que
%          hace que LaTeX ignore directamente todo lo que está adentro
\usepackage{verbatim}
%
% moreverb implementa el entorno verbatimtab indentando los tabs que encuentre,
%          y también el entorno listing, que pone números de línea al verbatim.
%          Para cambiar el ancho de la tabulacion, uso
%          \renewcommand\verbatimtabsize{<ancho del tab>\relax}
%          También define el entorno boxedverbatim.
\usepackage{moreverb}
%
% listings me da el entorno lstlisting con resaltado de sintaxis.
%          Para setear el lenguaje del código, hago \lstset{language=<lang>}
\usepackage{listings}
%
% url es un verbatim para escribir URL's que permite linebreaks dentro de ésta.
%     para usarlo, \url{<URL>}
\usepackage{url}
%
% ===
% === Más packages
% ===
%
\usepackage{mdwlist}		%Para listas mas compactas
\usepackage{textcomp}		%Para algunos símbolos
\usepackage{colortbl}		%Para celdas de colores en tablas
\usepackage{fancyhdr}		%Para encabezados/pie
\usepackage{bbold}		%Fuente bb para modo math: \mathbb{R} = reales
\usepackage{dsfont}		%Fuente ds para modo math: \mathds{R} = reales
\usepackage{multirow}		%Para "combinar" celdas en tablas
\usepackage{float}		%Para cuadros, figuras, etc copadas
\usepackage{fancybox}		%Para recuardos de texto con bordes "fancy"
\usepackage{dingbat}		%Para dingbats
%\usepackage{marginal}		%Para notas al margen que no puedo hacer andar
\usepackage{amsmath}		%Para enornos matemáticos mas flexibles
%\usepackage{varwidth}		%varwidth es un minipage que se ajusta al ancho mínimo


\include{beamerconf}
\usetheme{Bergen}
\usecolortheme{orchid}

\title{Auditoría Informática}
\subtitle{Trabajo Final}
\author{Romina V. Galarza \and M. Nicolás Mastaglia \and Mauro J. Torrez}
\date{\today}

\begin{document}


\frame{\selectlanguage{spanish}\titlepage}

\section{Introducción}

\begin{frame}{Introducción}
    \begin{itemize}
    \item Lugar elegido para la realización del TP:
    \end{itemize}
Rectorado de la Universidad Nacional del Litoral
\end{frame}


\begin{frame}{Desarrollo}
  \large Problemas, necesidades e incertidumbres existentes en la
  Organización. Propuestas de solución.
\end{frame}


\section[Outline]{Problemas y soluciones propuestas}

\subsection[Outline]{Enlaces de red}

\begin{frame}{Enlaces de red}
  \begin{itemize}
  \item Problema:
    \begin{itemize}
    \item No se encontraron enlaces redundantes en la red.
    \end{itemize}
  \item Riesgos:
    \begin{itemize}
    \item Dejar fuera de servicio a sectores del Rectorado.
    \end{itemize}
  \item Propuesta:
    \begin{itemize}
    \item Agregar enlaces redundantes a la red de área local
        \item Utilizar los enlaces preexistentes.
          \begin{itemize}
          \item Colocar enlaces nuevos: cobre, fibra, inalámbricos.
          \item Redundancia de enlaces y switches en la sala de
            servidores.
          \end{itemize}
        \item Consecuencia adicional: aumento de la velocidad en
          funcionamiento normal.
    \end{itemize}
  \end{itemize}
\end{frame}


\section[Outline]{Sala de servidores}

\begin{frame}{Sala de servidores}
  \begin{itemize}
  \item Problema:
    \begin{itemize}
    \item El espacio es reducido
    \end{itemize}
    \item Riesgos:
      \begin{itemize}
      \item Imposibilidad de expansión de la sala
      \end{itemize}
  \item Propuesta:
    \begin{itemize}
    \item Se evaluaron tres posiblidades:
      \begin{itemize}
      \item Reubicar el área de soporte técnico
      \item Realizar una reforma edilicia
      \item Mudar la sala de servidores
      \end{itemize}
    \end{itemize}
  \end{itemize}
\end{frame}


\begin{frame}{Sala de servidores}
  \begin{itemize}
  \item Problema:
    \begin{itemize}
    \item No existe un acondicionador de aire de ``back-up''
    \end{itemize}
    \item Riesgos:
      \begin{itemize}
      \item Interrupción del servicio
      \item Recalentamiento de equipos
      \end{itemize}
  \item Propuesta:
    \begin{itemize}
    \item Instalar otro equipo de aire acondicionado
    \end{itemize}
  \end{itemize}
\end{frame}


\section{Espacio de trabajo}

\begin{frame}{Espacio de trabajo}
  \begin{itemize}
    \item Problema:
      \begin{itemize}
      \item Se incrementó el número de puestos de trabajo
        manteniendose el mismo espacio físico
      \end{itemize}
    \item Propuesta:
      \begin{itemize}
        %% \item Reubicación de los recursos humanos de cada área:
        %%   \begin{enumerate}
        %%   \item Administración y Desarrollo.
        %%   \item Soporte Técnico.
        %%   \item Cómputos.
        %%   \end{enumerate}
        \item Reubicar a los empleados de soporte técnico y de cómputos
          hacia otras oficinas dado que son los menos numerosos.
      \end{itemize}
    \end{itemize}
\end{frame}


\section{Plan de contingencia eléctrico}

\begin{frame}{Seguridad eléctrica}
  \begin{itemize}
  \item Problemas:
    \begin{itemize}
    \item No existe un plan de contingencia ante fallas
      %en el suministro eléctrico
    \item UPS mal conectado
    \item Grupo electrógeno sin conectar
    \end{itemize}
  \item Riesgos:
    \begin{itemize}
    \item Pérdida de datos
    \item Daño a los equipos informáticos y al UPS
    \end{itemize}
  \item Propuesta:
    \begin{itemize}
    \item Plan de reducción de riesgos:
      \begin{itemize}
      \item Reformar el cableado para que el UPS alimente sólo el equipamiento informático
      \item Conectar el grupo electrógeno al circuito
      \end{itemize}
    \item Plan de contingencia: % que tenga las siguientes pautas:
      \begin{itemize}
      \item Actividades a realizar durante la falla
      \item Actividades a realizar después de la falla
      \item Evaluación de los resultados y retroalimentación %del plan de acción).
      \end{itemize}
    \end{itemize}
  \end{itemize}
\end{frame}


\section{Control de acceso físico}

\begin{frame}{Seguridad física}
  \begin{itemize}
  \item Problema:
    \begin{itemize}
    \item No existe control de acceso físico
    \end{itemize}
  \item Riesgos:
    \begin{itemize}
    \item Ingreso de personal no autrizado
    \item Robo de equipamiento
    \item Robo de información
    \end{itemize}
  \item Propuesta:
    \begin{itemize}
    \item Colocar cámaras de seguridad
    \item Colocar dispositivo de control de acceso a la sala de
      servidores.
    \end{itemize}
  \end{itemize}
\end{frame}


\section{Seguridad Lógica}

\begin{frame}{Seguridad lógica}
  \begin{itemize}
  \item Problema:
    \begin{itemize}
    \item Falta de autenticación en la red local
    \end{itemize}
  \item Riesgo:
    \begin{itemize}
    \item Cualquiera puede ganar acceso a la red solo conectando una
      PC
    \end{itemize}
  \item Propuesta:
    %% %% Investigar una posible incorporación de tecnología de
    %% autenticación mediante %% un servidor RADIUS a la red de área
    %% local. Esta no será una tarea menor, ya %% que se hace necesario
    %% configurar todos los equipos de la red, desde %% servidores hasta
    %% las impresoras en red, incluyendo las PCs de los puestos de %%
    %% trabajo y los switches.
    \begin{itemize}
    \item Evaluar factibilidad de implementación de tecnologías de
      autenticación (RADIUS, Kerberos, etc.)
    \item Diseñar un plan de implementación gradual
      %%  \begin{enumerate}
      %%  \item Comenzar por servicios que manejan información crítica
      %%  \item Seguir por los sistemas que manejan información
      %%    confidencial pero no crítica
      %%  \item Finalmente, incorporar al resto de los usuarios y el
      %%    equipamiento como impresoras de red, lo cual traería consigo
      %%    el beneficio de que todos los usuarios podrán estar seguros
      %%    que el entorno de red al que se conectan es fiable y
      %%    controlado.
      %% \end{enumerate}
    \end{itemize}
  \end{itemize}
\end{frame}


\begin{frame}{Seguridad lógica: servicio LDAP}
  \begin{itemize}
  \item Problema:
    \begin{itemize}
    \item Las claves de los usuarios se transportan en la red sin encriptación
    \end{itemize}
  \item Riesgo:
    \begin{itemize}
    \item Obtención de claves mediante \emph{sniffing}
    \end{itemize}
  \item Propuesta:
    \begin{itemize}
    \item Configurar la utilización del servicio LDAP a través de HTTP seguro (HTTPS)
    \end{itemize}
  \end{itemize}
\end{frame}


\begin{frame}{Seguridad lógica: servidor de correo}
  \begin{itemize}
  \item Problemas:
    \begin{itemize}
    \item Seguridad optativa en conexiones POP e IMAP
    \item Servidor de salida SMTP no autenticado
    \end{itemize}
  \item Riesgos:
    \begin{itemize}
    \item Obtención de claves mediante inspección de paquetes
    \item ``Toma'' del servidor de salida para envío masivo de correos basura
    \end{itemize}
  \item Propuesta:
    \begin{itemize}
    \item Incorporar autenticación al servidor de salida SMTP
    \item Establecer cuota máxima por usuario
    \item Inhabilitar las conexiones no seguras
    \item Desafío estilo CAPTCHA para la interfaz webmail
    \end{itemize}
  \end{itemize}
\end{frame}


\begin{frame}{Seguridad lógica: archivos compartidos}
  \begin{itemize}
  \item Problema:
    \begin{itemize}
    \item Los usuarios de Windows comparten archivos de forma insegura
    \end{itemize}
  \item Riesgo:
    \begin{itemize}
    \item Robo de información confidencial
    \end{itemize}
  \item Propuesta:
    \begin{itemize}
    \item Implementar un sistema centralizado de manejo de documentos
    \item Capacitar a los usuarios para que utilicen el sistema
    \end{itemize}
  \end{itemize}
\end{frame}


\section{Proyectos abortados}
\begin{frame}
  \begin{itemize}
  \item Problema:
    \begin{itemize}
    \item Muchos proyectos se ``congelan'' y se terminan abortando
    \end{itemize}
  \item Riesgos:
    \begin{itemize}
    \item Horas de trabajo invertidas no productivas
    \end{itemize}
  \item Propuesta:
    \begin{itemize}
    \item Sistema de seguimiento de proyectos:
      \begin{enumerate}
      \item Un plan de tareas con fechas estimadas de concreción
      \item Alarmas de incumplimiento
      \end{enumerate}
    \item Manual de procedimiento para proyectos nuevos
    \item Incorporar personal a Soporte Técnico y Administración
    \item Evaluar estado actual y viabilidad de los proyectos:
      \begin{itemize}
      \item Puppet
      \item Servidor de correo
      \item Capacitación a los usuarios
      \item Proyecto ``/oficina''
      \end{itemize}
    \end{itemize}
  \end{itemize}
\end{frame}


\section{Mantenimiento en PCs de los usuarios}
\begin{frame}{Mantenimiento en las PCs de usuarios}
  \begin{itemize}
  \item Problema:
    \begin{itemize}
    \item Falta mantenimiento en las PCs de los usuarios
    \end{itemize}
  \item Riesgos:
    \begin{itemize}
    \item Administrabilidad reducida
    \item Problemas de compatibilidad de programas
    \item Mayor demanda de soporte técnico
    \end{itemize}
  \item Propuesta:
    \begin{itemize}
    \item Actualizar y estandarizar configuración en los equipos
    \item Reactivar el proyecto Puppet, o encarar uno nuevo con
      similares objetivos
    \end{itemize}
  \end{itemize}
\end{frame}


\section{Relaciones comerciales}
\begin{frame}{Relaciones comerciales}
  \begin{itemize}
  \item Problema:
    \begin{itemize}
    \item Una única persona maneja las relaciones con proveedores
    \end{itemize}
  \item Riesgo:
    \begin{itemize}
    \item Demora en caso de ausencia
    \end{itemize}  
  \item Propuesta:
    \begin{itemize}
    \item Documentar operaciones efectuadas
    \item Mantener un registro con datos de los proveedores
    \item Asignar acceso a estos datos a una o dos personas más
    \end{itemize}
  \end{itemize}
\end{frame}


\section{Conclusión}
\begin{frame}{Conclusión}
  \begin{itemize}
  \item Se llevan a cabo buenas prácticas en lo que refiere a la seguridad de
    los sistemas, aunque se han encontrado algunos puntos débiles.
  \item Los problemas mayores son la insuficiencia de espacio físico, tanto
    para la sala de servidores como para el personal.
  \item Otro de los puntos generales de los problemas es la falta de normativa
    en los procedimientos, así como la escasa capacitación.
  %% \item Se sugiere llevar a cabo las tareas propuestas en el Plan de Mejoras
  %%   para lograr satisfacer las demandas actuales de los usuarios y reducir el
  %%   riesgo en lo que a respecta a la seguridad de los sistemas informáticos.
  \end{itemize}
\end{frame}
\end{document}




















